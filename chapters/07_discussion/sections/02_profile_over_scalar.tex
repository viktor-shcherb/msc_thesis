\section{Why the Profile Matters More Than the Scalar}
\label{sec:profile-over-scalar}

Aggregate plasticity ($\text{AP}_{\text{overall}}$) fails as a cross-family
predictor of long-context performance. The positional degradation
pattern---$\text{AP}_{\text{drop}}$---succeeds. This distinction has practical
implications for model evaluation.

\paragraph{The Ministral-3B diagnostic.}
Ministral-3-3B is the diagnostic outlier
(Figure~\ref{fig:apdrop-vs-lbp}): it has the highest aggregate plasticity
(0.622) among all 7 models with LongBench-Pro scores, yet scores only 30.18.
Its plasticity profile is nearly flat ($\text{AP}_{\text{drop}} = 0.072$,
comparable to the 14B variant), indicating strong context preservation. The
low benchmark score reflects limited base capability at 3B scale---the model's
knowledge and reasoning ability, not its attention mechanics, is the
bottleneck.

\paragraph{Decomposing benchmark performance.}
Long-context benchmark performance depends on two factors: \emph{base model
capability} (knowledge, reasoning, instruction following) and \emph{context
preservation} (maintaining attention flexibility at distance). Aggregate
plasticity approximates the second factor, but since context preservation
varies less than base capability across model scales, the aggregate is
dominated by the noisier capability term in cross-family comparisons.

$\text{AP}_{\text{drop}}$ isolates context preservation more cleanly. By
measuring the \emph{slope} of plasticity decline rather than the absolute
level, it is robust to differences in base capability. Within a family
(controlled architecture and training recipe), both $\text{AP}_{\text{drop}}$
and LBP move together---larger Qwen-3 models degrade less and score higher.
Across families, $\text{AP}_{\text{drop}}$ separates families by context
preservation strategy while remaining agnostic to base capability.

\paragraph{Practical implication.}
When evaluating a model for long-context deployment, aggregate benchmark
scores conflate context preservation with base capability. A model with low
aggregate LBP but flat $\text{AP}_{\text{drop}}$ is
\emph{capability-limited}---it preserves context well but lacks the knowledge
or reasoning ability to exploit it. A model with high aggregate LBP but
steep $\text{AP}_{\text{drop}}$ is \emph{context-limited}---it performs well
on average but may fail unpredictably on inputs requiring information from
distant positions. Per-position plasticity profiles provide a more targeted
diagnostic than aggregate scores for distinguishing these failure modes.
