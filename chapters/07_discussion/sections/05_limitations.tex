\section{Limitations}
\label{sec:limitations}

\paragraph{Observational design.}
All analyses are observational: we measure correlations between mechanistic
metrics and behavioral performance, not causal effects. Claiming that low
plasticity \emph{causes} ECL failure would require interventions---ablating
low-plasticity heads and measuring retrieval accuracy degradation. We make
associative claims only.

\paragraph{Single capture dataset.}
All analyses are computed over a fixed set of 500 LongBench-Pro examples at
128K+ length. The stability of plasticity profiles across document types
(code, dialogue, structured data) is untested. The profiles may partly reflect
dataset-specific statistics rather than purely architectural properties.

\paragraph{Three model families.}
Eleven models from three families (Ministral-3, Qwen-3, Llama-3.2) is a
limited sample. The findings may not generalize to architecturally different
models such as mixture-of-experts models~\cite{2024-05-deepseek-v2-moe,
2024-12-deepseek-v3-technical-report}, models with linear
attention~\cite{2024-05-mamba-selective-state-spaces,
2024-10-rwkv-eagle-finch-matrix-states}, or models at substantially larger
scales. The training dynamics story relies on a single model's
trajectory (SmolLM3-3B).

\paragraph{Pairwise ranking, not softmax weights.}
Plasticity evaluates attention as a reranking mechanism: it measures whether
query content determines the pairwise ordering of keys. Since any total
ordering is determined by its pairwise comparisons, corrupted pairwise
orderings imply a corrupted global ranking. However, plasticity does not
directly measure softmax weight concentration. A head could produce correct
pairwise orderings yet still spread attention too thinly across many keys, or
conversely produce incorrect orderings that are masked by softmax saturation.
The metric captures ranking quality, not weight allocation.

\paragraph{Linear drift and Gaussian assumptions.}
The plasticity decay theorem (Theorem~\ref{thm:plasticity-decay}) and the
Gaussian closed form (Equations~\ref{eq:gaussian-mean}--\ref{eq:plasticity-gaussian})
assume that positional drift is linear and that score differences are
approximately Gaussian. Both are empirically supported---linear fits achieve
high $R^2$ on the drift axis, and Q-Q plots of score differences show
approximate normality---but they are approximations. Non-linear positional
structure (e.g., from RoPE's rotation planes in the complement subspace) is
not captured by the linear model and may contribute to plasticity decay
through mechanisms the theorem does not describe. The Gaussian assumption
may also break down for heads with strongly non-Gaussian content
distributions, though the central limit theorem provides some robustness
when aggregating over many key pairs.

\paragraph{Per-head independence.}
We analyze each attention head independently. Per-head plasticity does not
capture cross-layer interactions: a low-plasticity early head feeding into a
high-plasticity later head may not limit model-level ECL. Circuit-level
analysis~\cite{2021-12-transformer-circuits-framework,
2022-03-in-context-learning-induction-heads} would be needed to understand how
individual head profiles compose into model-level behavior.

\paragraph{Base vs.\ instruct model confound.}
Mechanistic metrics are computed on base (non-instruct) model weights, while
benchmark scores come from instruct-tuned variants of the same architectures.
Instruction tuning could alter the attention geometry we measure, in which case
the base-model metrics may not fully correspond to the instruct-model behavior
on benchmarks. The correlation between plasticity drop and benchmark ordering
is therefore an association between base-model geometry and instruct-model
performance, not a direct measurement of the mechanism underlying the benchmark
scores.

\paragraph{GQA key sharing.}
In GQA models, multiple query heads share the same key head. We analyze each
(query head, shared key head) pair independently but do not study whether
query heads within a GQA group develop coordinated or divergent plasticity
profiles. Coordination within GQA groups could amplify or mitigate the effects
we observe at the individual head level.
