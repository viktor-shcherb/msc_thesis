\section{Summary}
Behavioral studies provide robust evidence that effective context can be far
below claimed context and that estimates are sensitive to benchmark design.
Mechanistic studies identify candidate causes, including position-related
effects and retrieval-path fragility. Context-extension methods improve usable
length in many settings but do not provide a mechanistic quantification
framework for
ECL~\cite{2024-10-ruler-context-size,2025-07-nolima-long-context-evaluation,2026-01-longbench-pro,2025-04-attention-sink-emerges,2025-04-retrieval-head-long-context-factuality,2023-06-pi-positional-interpolation,2024-05-yarn-context-extension}.

The next chapter develops the geometric framework that connects these
behavioral and mechanistic perspectives: it extracts positional bias and
attention plasticity from internal representations and provides metrics
that can be correlated with behavioral long-context performance.
