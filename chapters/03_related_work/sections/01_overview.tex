\section{Overview}

This chapter reviews prior work on effective context length (ECL), with emphasis
on evidence quality and mechanistic interpretability. The literature can be
organized into three strands: behavioral evaluation of long-context capability,
mechanistic analyses of long-context failure, and context-extension or
mitigation methods
\cite{2024-10-ruler-context-size,2025-07-nolima-long-context-evaluation,2025-04-attention-sink-emerges,2023-06-pi-positional-interpolation,2024-05-yarn-context-extension}.
% Verification (three-strand framing): verify benchmark-centric analyses in
% RULER (2404.06654) and NoLiMa (2502.05167), mechanism-centric analyses in
% Attention Sink (2410.10781), and mitigation methods in PI (2306.15595) and
% YaRN (2309.00071).

The definitions of claimed and effective context length introduced in
Section~\ref{sec:effective-context-length} apply throughout this chapter.

The chapter first synthesizes behavioral evidence, then reviews mechanistic
findings, then surveys mitigation families, and finally positions the research
gap addressed by this thesis.
% Verification (chapter roadmap): confirm ordering in this chapter's section
% sequence.
