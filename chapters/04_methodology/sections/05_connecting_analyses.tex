\section{Connecting the Three Analyses}
\label{sec:connecting-analyses}

The three analyses are not successive levels of abstraction. They are three
complementary questions about the same Q/K geometry, each providing information
the others cannot. Table~\ref{tab:analysis-summary} summarizes the framework.

\begin{table}[t]
\centering
\small
\caption{Summary of the three-analysis framework. Each analysis operates on the
same captured Q/K vectors and answers a distinct question about the head's
embedding geometry.}
\label{tab:analysis-summary}
\begin{tabular}{@{}llll@{}}
\toprule
Analysis & Computes & Question & Key metric \\
\midrule
PCA & Variance decomposition, & How is Q/K variance & $r_q$, $r_k$ on PC0; \\
    & position correlation    & distributed?        & head taxonomy \\
\addlinespace
Rotation & Targeted axis & What is the parametric & bias\_strength \\
         & projections, drift slopes & form of positional bias? & $= \mu_Q^a \times \alpha_K$ \\
\addlinespace
Plasticity & $\Pr$(random query & Does positional bias & $\text{AP}_{\text{drop}}$ \\
           & flips key ordering) & dominate content? & \\
\bottomrule
\end{tabular}
\end{table}

Two aspects of the inter-analysis relationships deserve emphasis.

\paragraph{Two related orthogonal transformations.}
The rotation analysis (Section~\ref{sec:planar-rotation-model}) and the
plasticity framework (Section~\ref{sec:attention-plasticity}) both use
orthogonal transformations that preserve $q^\top k$, but they differ in
construction and purpose. The rotation constructs axes from the combined Q+K
pool---axis $a$ for shared positional drift, axis $b$ for Q/K
separation---and provides geometric characterization of the bias mechanism. The
plasticity framework uses a query-only Householder reflection---coordinate~1
for query positional drift---to enable the formal decay theorem and Gaussian
closed form. On the drift axis, both transformations isolate position
covariance, and the resulting projections are closely related. The rotation
additionally constructs axis $b$ (Q/K separation), which the plasticity
framework does not need.

\paragraph{Logical progression.}
The analyses address progressively deeper questions. PCA discovers that position
structure exists and is pervasive. The rotation isolates and parameterizes the
bias mechanism, yielding bias strength as a scalar summary. Plasticity tests
whether the bias functionally constrains attention, accounting for the content
signal that may compensate.

When plasticity is low, the rotation analysis tells us \emph{why} (large bias
strength). When bias strength is large, plasticity tells us \emph{whether it
matters} (whether content can compensate). This complementarity motivates
reporting all three analyses: a head may have large bias strength yet high
plasticity (content dominates despite the bias), or small bias strength yet low
plasticity (content signal is too weak to drive flexible retrieval even with
minimal positional preference). The results in
Chapter~\ref{chap:results} demonstrate both patterns across the model families
in our study.
